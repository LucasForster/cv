\cvsection{Education}

\cvevent%
    {\en{Computer Science}\de{Informatik} (B.~Sc.) \yearRight{Sep 2021}}
    {RWTH Aachen\en{ University}}{}{}
%
\en{Overall mark 2.6}
\de{Gesamtnote 2.6}
\smallskip\\
\begin{itemize}
\item%
    \en{Mathematical and theoretical foundations}
    \de{Mathematische und theoretische Grundlagen}
\item%
    \en{Scientific research and writing}
    \de{Wissenschaftliches Recherchieren und Verfassen}
    %
    \begin{itemize}
    \item%
        \en{Low-level logic programming}
        \de{Low-Level-Programmierung}
    \item%
        \en{Computer graphics}
        \de{Computergraphik}
    \end{itemize}
\item%
    \en{Practical software courses (in teams)}
    \de{Angewandte Softwareprojekte (im Team)}
    %
    \begin{itemize}
    \item%
        \en{Board Game AI}
        \de{Brettspiel-KI}
    \item%
        \en{Microcontroller operating system}
        \de{Mikrokontrollerbetriebssystem}
    \end{itemize}
\item%
    \en{Elective courses}
    \de{Wahlfächer}
    \begin{itemize}
    \item%
        \en{Efficient Algorithms}
        \de{Effiziente Algorithmen}
    \item%
        \en{Artifical Intelligence}
        \de{Künstliche Intelligenz}
    \item%
        \en{Automata Theory}
        \de{Automatentheorie}
    \item%
        \en{Technical English}
        \de{Technisches Englisch}
    \end{itemize}
\item%
    \en{Physics as application subject}
    \de{Physik als Anwendungsfach}
\end{itemize}
\smallskip
\en{Final thesis on a topic in Intelligent Mobility:}
\de{Abschlussarbeit zu einem Thema aus Intelligenter Mobilität (englisch):}
\\\smallskip
\textit{Disaggregating Origin-Destination Matrices
using Time-Progressive Graphs for Agent-Based Traffic Simulations}
\\\smallskip
\en{Published abridged version:}
\de{Veröffentlichte Kurzversion:}
\\
\link{doi.org/10.1016/j.procs.2022.03.072}%

\divider

\cvevent%
    {Abitur \yearRight{2015}}
    {Max-Planck Gymnasium Saarlouis}{}{}
%
\en{Overall mark 2.1}
\de{Gesamtnote 2.1}
\smallskip\\
\begin{itemize}
\item%
    \en{Computer science as main subject from 8th grade onwards}
    \de{Informatik als Hauptfach ab der 8. Klasse}
\item%
    \en{Working group Artificial Intelligence}
    \de{Arbeitsgemeinschaft Künstliche Intelligenz}
\item%
    \en{Final exams in mathematics, computer science, French, German, history}
    \de{Abschlussprüfungen in Mathematik, Informatik, Französisch, Deutsch, Geschichte}
\end{itemize}%

\divider

\cvevent%
    {\en{French Primary School}\de{Französische Grundschule}}
    {Institut de la Providence}{}{}
